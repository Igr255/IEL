\section{\Large Príklad 2F}
{\Large Zadané hodnoty:}
{\Large

$U$ = 130V

$R_1$ = 180$\Omega$, $R_2$ = 350$\Omega$, $R_3$ = 600$\Omega$,

$R_4$ = 195$\Omega$, $R_5$ = 650$\Omega$, $R_6$ = 250$\Omega$}

\begin{center}
\begin{circuitikz} \draw


(0,5) to [dcvsource] (0,3) -- (7,3) --(7,5)
      to [generic, a=$R_5$, *-] (7,8)
      to [generic, a=$R_4$] (4,8)
      to [generic, a=$R_1$, *-] (1,8) -- (1,5)
      to [generic, a=$R_2$, *-*] (4,5)
      to [generic, a=$R_6$] (7,5)
(0,5) -- (1,5)
(4, 8) to [generic, a=$R_3$] (4,5);

\draw (4,7.8) [very thick,->] node[left] {$I_{R3}$} -- (4,7.2);
\draw (-0.7,4.5) [->] node[left] {$U_{EKV}$} -- (-0.7,3.5);
\draw (4.5,7) [->] node[right] {$U_{R_3}$} -- (4.5,6);
\end{circuitikz}
\end{center}
\bigskip

{\Large Odvodenie a výpočet $R_{TH}$} 
\bigskip

{\large Odstránime hľadaný rezistor a skratujeme zdroj:}

\begin{center}
\begin{circuitikz} \draw


(1,5) -- (1,3) -- (7,3) --(7,5)
      to [generic, a=$R_5$, *-] (7,7)
      to [generic, a=$R_4$] (4,7)
      to [generic, a=$R_1$, *-] (1,7) -- (1,5)
      to [generic, a=$R_2$, *-*] (4,5)
      to [generic, a=$R_6$] (7,5)
(4, 7) to [short, *-o] (4,6.5)
(4, 5) to [short, *-o] (4,5.5);

\end{circuitikz}
\end{center}

\newpage
{\large Následne obvod upravíme:}

\begin{center}
\begin{circuitikz} \draw

(7,5) -- (7,7)
      to [generic, a=$R_{45}$] (4,7)
      to [generic, a=$R_1$, *-] (1,7) -- (1,5)
      to [generic, a=$R_2$, -*] (4,5)
      to [generic, a=$R_6$] (7,5)
(4, 7) to [short, *-o] (4,7.5)
(4, 5) to [short, *-o] (4,4.5)
(1,6)  -- (7,6) 
(1, 6) to [short, *-*] (1,6)
(7, 6) to [short, *-*] (7,6);

\end{circuitikz}
\end{center}
{\large\[ R_{45} = R_4 + R_5 = 845\Omega\]}
\bigskip
\begin{center}
\begin{circuitikz} \draw

(7,5) -- (7,7)
      to [generic, a=$R_1$] (4,7)
      to [generic, a=$R_2$, *-] (1,7) -- (1,5)
      to [generic, a=$R_6$, -*] (4,5)
      to [generic, a=$R_{45}$] (7,5)
(4, 7) -- (4, 5)

(0.5, 6) to [short, o-*] (1,6)
(7, 6) to [short, *-o] (7.5,6);

\end{circuitikz}
\end{center}
{\large\[ R_{145} = \frac{R_{45} \times R_1}{R_{45} + R_1} = \frac{845 \times 180}{1025} = \frac{6084}{41}\Omega\]}
{\large\[ R_{26} = \frac{R_2 \times R_6}{R_2 + R_6} = \frac{350 \times 250}{600} = \frac{875}{6}\Omega\]}
\bigskip

{\large Vypočítame hodnotu $R_{TH}$:} 
\bigskip

{\large\[ R_{TH} = R_{145} + R_{26} = \frac{875}{6} + \frac{6084}{41} = \frac{72379}{246}\Omega \]}

\newpage
{\Large Odvodenie a výpočet $U_{TH}$} 
\bigskip

{\large Namiesto $R_3$ dosadíme fiktívny napäťový zdroj a pomocou} 

{\large slučky $I_{TH}$ vypočítame $U_{TH}$:} 

\begin{center}
\begin{circuitikz} \draw


(1,5) to [dcvsource] (1,3) -- (7,3) --(7,5) -- (7,7)
      to [generic, a=$R_{45}$] (4,7)
      to [generic, a=$R_1$, *-] (1,7) -- (1,5) 
      to [generic, a=$R_2$, *-*] (4,5)
      to [generic, a=$R_6$,, -*] (7,5)
(4, 7) to [dcvsource] (4, 5); 
\draw[->,shift={(2.5,6)}] (120:.4cm) arc (120:-90:.4cm) node at(0,0){$I_{TH}$};
\draw (0.3,4.5) [->] node[left] {$U_{EKV}$} -- (0.3,3.5);
\end{circuitikz}
\end{center}
\bigskip

{\large Najprv vypočítame $U_{R_1}$ a $U_{R_2}$:} 

{\large\[ U_{R_1} = U_{EKV} \times \frac{R_1}{R_1 + R_{45}} = 130 \times \frac{180}{1025} = \frac{936}{41}V\]}
{\large\[ U_{R_2} = U_{EKV} \times \frac{R_2}{R_2 + R_6} = 130 \times \frac{350}{600} = \frac{455}{6}V \]}
\bigskip

{\large Vypočítame $U_{TH}$:} 
{\large\[ U_{TH} = U_{R_2} - U_{R_1} = \frac{455}{6} - \frac{936}{41} = \frac{13039}{246}V \]}
\bigskip

{\Large Vypočítané hodnoty dosadíme do náhradnej schémy:} 

\begin{center}
\begin{circuitikz} \draw


(1,5) to [dcvsource] (1,3) -- (3, 3)
      to [generic, a=$R_3$] (3,5)
      to [generic, a=$R_{TH}$] (1,5);

\draw (0.3,4.5) [->] node[left] {$U_{EKV}$} -- (0.3,3.5);
\end{circuitikz}
\end{center}
\bigskip

{\large Vypočítame $I_{R_3}$ a $U_{R_3}$:} 
{\large\[ I_{R_3} = \frac{U_{TH}}{R_{TH} + R_3} = \frac{\frac{13039}{246}}{\frac{72379}{246} \times 600} \doteq 59,2738 mA\]}
{\large\[ U_{R_3} = I_{R_3} \times R_3 = 0,0592738 \times 600 \doteq 35,5643V\]}